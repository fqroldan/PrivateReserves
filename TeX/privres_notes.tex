\documentclass[11pt,letterpaper]{article}
\usepackage[american]{babel}
\usepackage{amssymb,amsfonts,amsmath,amsthm,amscd,verbatim,latexsym,xspace,mathtools,array,pifont}
%\usepackage[sfdefault]{FiraSans}
\usepackage{ifxetex}
\ifxetex
	\usepackage[math]{mathspec}
	\setmainfont[Numbers=OldStyle]{Linux Libertine}
	\setmathsfont(Latin,Digits){Linux Libertine}
	\setlength{\parskip}{0.15cm}
	\linespread{1.3} % Line spacing
\else
	\usepackage[osf]{mathpazo}
	\usepackage{palatino}
\fi

% \setmainfont[ItalicFont	= Palatino Linotype Italic,
% 			 BoldFont	= Palatino Linotype Bold]
% 	{Palatino}

% \setmainfont[SmallCapsFont	= Fontin SmallCaps]{TeX Gyre Pagella}
% \setmainfont
%      {STIX Two Text}
	% \usepackage{mathspec}
	% \setmathsfont(Digits,Latin,Greek)[Numbers={Lining,Proportional}]{STIX Two Math}
	%
	% \let\oldtextsc\textsc
	% \renewcommand{\textsc}[1]{{\setmainfont[Numbers=OldStyle]{STIX Two Text} \scshape #1}}
	% \let\oldscshape\scshape
	% \renewcommand{\scshape}{\setmainfont[Numbers=OldStyle]{STIX Two Text} \oldscshape}

\usepackage{booktabs}
\usepackage{datetime}
\usepackage{bbm} %for the sweet indicator function
\usepackage[usenames,dvipsnames]{xcolor}
\usepackage{enumerate}
\usepackage{graphicx}
\usepackage{etoolbox}
\usepackage{listings} % Required for insertion of code
%\usepackage{courier} % Required for the courier font
\usepackage{makeidx}
\usepackage[left=2.0cm,right=2.0cm,top=1in,bottom=1in,nomarginpar,bindingoffset=0cm,centering]{geometry}
%\usepackage{paralist} % Used for the compactitem environment which makes bullet points with less space between them
\usepackage{caption}
\usepackage{subcaption}
\captionsetup[table]{labelfont=sc,textfont=sc}
\captionsetup[figure]{labelfont=sc,textfont=sc}

\usepackage{titlesec} % Allows customization of titles
%\renewcommand\thesection{\Roman{section}} % Roman numerals for the sections
%\renewcommand\thesubsection{\Alph{subsection}} % Roman numerals for subsections
\titleformat{\section}[block]{\large\scshape\centering}{\thesection.}{1em}{} % Change the look of the section titles
\titleformat{\subsection}[block]{\large\itshape}{\thesubsection\xspace}{1em}{} % Change the look of the section titles

\usepackage{fancyhdr} % Custom headers and footers
\pagestyle{fancyplain} % Makes all pages in the document conform to the custom headers and footers
\fancyhead{} % No page header - if you want one, create it in the same way as the footers below
\fancyfoot[L]{} % Empty left footer
\fancyfoot[C]{} % Empty center footer
\fancyfoot[R]{\thepage} % Page numbering for right footer
\renewcommand{\headrulewidth}{0pt} % Remove header underlines
\renewcommand{\footrulewidth}{0pt} % Remove footer underlines
\setlength{\headheight}{10pt} % Customize the height of the header
\usepackage[authoryear]{natbib}

%\numberwithin{equation}{section}
%\numberwithin{table}{section}
\definecolor{nyupurple}{HTML}{57068C}
\definecolor{imfblue}{HTML}{004c97}
\definecolor{DarkTeal}{HTML}{08282e}

\newcommand{\dem}[2] {\bigskip \indent \textbf{Proof of #1 }#2 \hfill $\square$}
\def\gt{>}
\def\lt{<}
\newcommand{\cmark}{\ding{51}}%
\newcommand{\xmark}{\ding{55}}%
\def\w{\omega}
\def\f{\varphi}
\def\s{\sigma}
\def\g{\gamma}
\def\subjectto{\quad \text{ subject to } \quad}
\newcommand{\real}{\mathbb{R}}
\newcommand{\nat}{\mathbb{N}}
\newcommand{\der}[2]{\frac{\partial #1}{\partial #2}}
\newcommand{\pfrac}[2]{\left( \frac{#1}{#2} \right)}
\newcommand{\prob}[2][{}]{\mathbb{P}_{#1}\left( #2 \right)}
\newcommand{\ex}[2][{}]{\mathbb{E}_{#1}\left[ #2 \right]}
\newcommand{\var}[2][{}]{\text{var}_{#1}\left( #2 \right)}
\newcommand{\ent}[1]{\text{ent}\left( #1 \right)}
\newcommand{\alert}[1]{{\color{red!75!black}#1}}
\newcommand{\note}[1]{\emph{Note:} {\color{red!60!black}#1}}
\newcommand{\noopsort}[2]{#2}
\newcommand{\intro}[1][{Introduction}]{\addtocounter{section}{+1} \section*{{#1}}}
\newcommand{\red}[1]{{\color{red} #1}}
\def\rd{R\&D\xspace}
\def\ind{\mathbbm{1}}
\def\inv{{-1}}
\def\mc{\mathtt{mc}}
\def\NFA{\mathtt{NFA}}
\def\nfa{\mathtt{nfa}}
\def\haircut{\hslash}
\newcommand{\horrule}[1]{\rule{\linewidth}{#1}}


\def\a{\alpha}
\def\b{\beta}
\def\de{\delta}
\def\De{\Delta}
\def\g{\gamma}
\def\G{\Gamma}
\def\eps{\varepsilon}
\def\ups{\upsilon}
\def\Ups{\Upsilon}
\def\sg{\sigma}
\def\Sg{\Sigma}
\def\lm{\lambda}
\def\Lm{\Lambda}
\def\th{\theta}
\def\Th{\Theta}
\def\om{\omega}
\def\Om{\Omega}
\def\vphi{\varphi}

\def\re{\mathbb{R}} %real numbers
\def\intg{\mathbb{Z}} %integers
\def\prob{\mathbb{P}} %probability
\def\e{\mathbb{E}} %expectation
\def\n{\mathbb{N}} %natural numbers
\def\ind{\mathbbm{1}} %sweet indicator function
\def\pref{\succsim} %preference relation "at least as good as"
\def\spref{\succ} % strict preference relation
\def\notpref{\precsim}
\def\snotpref{\prec}
\def\indiff{\sim} %indifference/equivalence relation/distributed as

\def\mb{\mathbb}
\def\mbm{\mathbbm}
\def\mbf{\mathbf}
\def\tbf{\textbf}
\def\mc{\mathcal}
\def\d{\displaystyle}
\def\dsum{\displaystyle\sum}
\def\dprod{\displaystyle\prod}
\def\dsup{\displaystyle\sup}
\def\dinf{\displaystyle\inf}
\def\dlim{\displaystyle\lim}
\def\dliminf{\displaystyle\liminf}
\def\dlimsup{\displaystyle\limsup}
\def\oline{\overline}
\def\ol{\overline}
\def\uline{\underline}
\def\ul{\underline}
\def\dcup{\displaystyle\bigcup}
\def\dcap{\displaystyle\bigcap}
\def\dmax{\displaystyle\max}
\def\dmin{\displaystyle\min}
\def\l{\left}
\def\r{\right}
\def\la{\langle}
\def\ra{\rangle}
\def\pr{\operatorname{Pr}}
\def\var{\operatorname{Var}}
\def\cov{\operatorname{Cov}}
\newcommand*{\argmax}{\operatornamewithlimits{argmax}\limits}
\newcommand*{\argmin}{\operatornamewithlimits{argmin}\limits}
%\def\argmin{\operatorname{argmin}}
%\def\argmax{\mathop{\argmax}}
%\def\dargmax{\displaystyle\argmax}
\def\cond{ \ \big| \ }
\def\w{\wedge}
\def\s1n{\sum_{i=1}^n}
\def\ds1n{\dsum_{i=1}^n}
\def\fr{\frac}
\def\goesas{\xrightarrow}
\def\goes{\rightarrow}
\def\gogo{\rightrightarrows}
\def\borel{\mc{B}(\re)}
\def\boreln{\mc{B}(\re^n)}
\def\intsc{\cap}
\def\dintsc{\dcap}
\def\union{\cup}
\def\dunion{\dcup}
\def\land{\wedge} %logical and
\def\lor{\vee} %logical or

%because \underaccent is defined while \overaccent is not...
\newcommand{\overaccent}[2]{\ensuremath{\accentset{\text{\tiny #1}}{\text{#2}}}}
%!!! use math mode inside

\def\sign{\operatorname{sign}}
\def\dom{\operatorname{dom}} %domain
\def\dim{\operatorname{dim}}
\def\spn{\operatorname{span}}
\def\det{\operatorname{det}}
\def\co{\operatorname{co}}
\def\aff{\operatorname{aff}}
\def\ri{\operatorname{ri}}
\def\cav{\operatorname{cav}} %concavification, yeah
\def\vex{\operatorname{vex}} %convexification, yeah

%\def\bah{\begin{align*}}
%\def\maamu{\end{align*}}
\def\bthm{\begin{theorem}}
\def\ethm{\end{theorem}}
\def\bdefn{\begin{definition}}
\def\edefn{\end{definition}}
\def\bprop{\begin{proposition}}
\def\eprop{\end{proposition}}
\def\bprob{\begin{problem}}
\def\eprob{\end{problem}}
\def\bprf{\begin{proof}}
\def\eprf{\end{proof}}
\def\bcor{\begin{corollary}}
\def\ecor{\end{corollary}}
\def\bconj{\begin{conjecture}}
\def\econj{\end{conjecture}}
\def\bexer{\begin{exercise}}
\def\eexer{\end{exercise}}
\def\bexam{\begin{example}}
\def\eexam{$\hfill\blacksquare$\end{example}}
\def\bnot{\begin{notation}}
\def\enot{\end{notation}}
\def\blemma{\begin{lemma}}
\def\elemma{\end{lemma}}
\def\bclaim{\begin{claim}}
\def\eclaim{\end{claim}}
\def\brem{\begin{remark}}
\def\erem{\end{remark}}
\def\bremark{\begin{remark}}
\def\eremark{\end{remark}}
\def\bfact{\begin{fact}}
\def\efact{\end{fact}}
\def\bass{\begin{assumption}}
\def\eass{\end{assumption}}

\usepackage{tikz}
\usetikzlibrary{shapes,arrows}
\tikzstyle{block} = [rectangle, draw, fill=blue!20, 
    text width=10em, text centered, rounded corners, minimum height=4em]
\tikzstyle{line} = [draw, -latex']
\usetikzlibrary{arrows, decorations.markings}
      \tikzstyle{vecArrow} = [thick, decoration={markings,mark=at position
      1 with {\arrow[semithick]{open triangle 60}}},
      double distance=1.4pt, shorten >= 5.5pt,
      preaction = {decorate},
      postaction = {draw,line width=1.4pt, white,shorten >= 4.5pt}]
     \tikzstyle{innerWhite} = [semithick, white,line width=1.4pt, shorten >= 4.5pt]
\usetikzlibrary{positioning}

\newcommand*\mycirc[1]{%
	\begin{tikzpicture}[baseline=(C.base)]
		\node[draw,circle,inner sep=1pt](C) {#1};
	\end{tikzpicture}}

\newtheorem{lemma}{Lemma}

\usepackage[pdfstartview=FitH]{hyperref}
\hypersetup{
    colorlinks	= true,
    linkcolor	= blue!75!black,
    urlcolor	= blue!50!black,
    citecolor 	= blue!75!black
}


\makeatletter
\patchcmd{\@bibitem}{\ignorespaces}{\label{bib-#1}\ignorespaces}{}{}\makeatother
\newcommand{\mycite}[1]{[\ref{bib-#1}]}
\newcommand{\myqcite}[2]{[\ref{bib-#2}, #1]}
\newcommand{\mytcite}[1]{\texorpdfstring{\cite{#1}}{\mycite{#1}}}
\newcommand{\mytqcite}[2]{\texorpdfstring{\citep[#1][]{#2}}{\myqcite{#1}{#2}}}
\linespread{1.32} % Line spacing

\def\acknowledgements{}

\title{Reserve Accumulation with a Private IIP\thanks{The views expressed herein are those of the authors and should not be attributed to the IMF, its Executive Board, or its management. \acknowledgements}}

\author{\href{https://fqroldan.github.io}{Francisco Rold\'{a}n}\thanks{e-mail: \href{mailto:froldan@imf.org}{\emph{froldan@imf.org}}}\\IMF \and \href{https://sosapadilla.github.io}{C\'{e}sar Sosa-Padilla}\thanks{e-mail: \href{mailto:csosapad@nd.edu}{\emph{csosapad@nd.edu}}} \\University of Notre Dame\\and NBER}

\ifdefined\ungated
\else
\def\ungated{1}
\fi

\date{\monthname\xspace \the\year %
% \\Preliminary: Click \href{https://fqroldan.github.io/resources/privres_draft.pdf}{here} for latest version%
}
\begin{document}

\ifnum\ungated=1%
	\maketitle 
	\begin{abstract}
		\noindent\input{abstract_privres.txt}
	\end{abstract}
	\vfill
	\noindent\textbf{JEL Classification}\quad F32, F34, F41\\
	\noindent\textbf{Keywords}\quad International reserves, sovereign default, macroeconomic stabilization, fixed exchange rates, inflation targeting.
	\bigskip
	\vfill
\else
	\includepdf[pages=-]{cover_imfwp.pdf}
	\tableofcontents
\fi
\pagebreak 

% \intro

\section{Model with government borrowing only}
\subsection{Model summary}
\paragraph{Government choices}

Main problem in repayment
\begin{align}
	V^R(b,a,\state) = \max_{b',a',c_T, h} &\; u(c_T, F(h)) + \beta \ex{\mathcal{V}\left(b',a',\state'\right)} \\
	\text{subject to }
	& c_T + q_a a' + \kappa b = a + y_T(\state) + q_b(b',a',\state) (b' - (1-\delta) b) \notag\\
	& h \leq \mathcal{H}(c_T, \bar{w}) \notag
\end{align}

Main problem in default
\begin{align}
	V^D(b,a,\state) = \max_{a',c_T, h} &\; u(c_T, F(h)) + \beta \ex{ \theta \mathcal{V}(b,a',\state') + (1-\theta) V^D(b,a',\state) } \\
	\text{subject to }
	& c_T + q_a a' = a + \mathcal{D}(y_T(\state)) \notag \\
	& h \leq \mathcal{H}(c_T, \bar{w}) \notag
\end{align}

Extreme value preference-for-default shocks yield default probabilities
\begin{align}
	\mathcal{P}(b,a,\state) = \frac{ \exp(V^D((1-\hslash)b,a,\state) / \sigma^V) }{\exp(V^D((1-\hslash)b,a,\state) / \sigma^V) + \exp(V^R(b,a,\state) / \sigma^V)}
\end{align}

Finally, the value of entering a period with access to markets (and hence an option to default) is
\begin{align}
	\mathcal{V}(b,a,\state) = (1-\mathcal{P}(b,a,\state)) V^R(b,a,\state) + \mathcal{P}(b,a,\state) V^D((1-\hslash)b,a,\state)
\end{align}

\paragraph{Foreigners, debt prices}
Stochastic discount factor
\begin{align}
	m(\state, \state') &= \exp(-r - \nu (\psi \epsilon' + 0.5 \psi^2\sigma_\epsilon^2)) \qquad \epsilon' = \log(y') - \rho\log(y) - (1-\rho)\mu_y\\
	q_a &= \exp(-r) \\
	q_b(b',a',\state) &= \ex{ m(\state,\state') \left(\ind_\mathcal{D'} (1-\hslash) q_b(b',a',\state') + (1-\ind_\mathcal{D'})(\kappa + (1-\delta)q_b(b',a',\state')) \right) }
\end{align}

\paragraph{Private economy}
Preferences
\begin{align}
	u(c_T,c_N) = \left[ \varpi_N c_N^{-\eta} + \varpi_T c_T^{-\eta} \right]^{-\frac{1}{\eta}} \qquad p_N = \frac{\varpi_N}{\varpi_T} \pfrac{c_T}{c_N}^{1+\eta}
\end{align}
for $\varpi_N + \varpi_T = 1$.

Production and default costs
\begin{align}
	y_N(h, \ind_\mathcal{D}) = (1 - \Delta \ind_\mathcal{D}) h^\alpha \qquad \mathcal{D}(y) = (1-\Delta) y
\end{align}

Wage rigidities
\begin{align}
	h \leq \mathcal{H}(c_T, \bar{w}, \ind_\mathcal{D}) \qquad\qquad \mathcal{H}(c_T, \bar{w}, \ind_\mathcal{D}) = \left(\frac{\varpi_N}{\varpi_T} (1 - \Delta \ind_\mathcal{D}) \frac{\alpha}{\bar{w}}\right)^\frac{1}{1+\alpha\eta} c_T^\frac{1+\eta}{1+\alpha\eta}
\end{align}


\subsection{Model results}

\begin{figure}[!hbtp]\centering
	\includegraphics[width=0.9\linewidth]{../Graphs/simul.pdf}
	\caption{A simulated path}
\end{figure}

\begin{figure}[!hbtp]\centering
	\includegraphics[width=0.9\linewidth]{../Graphs/vr-vd.pdf}
	\caption{Default incentives}
\end{figure}

\begin{figure}[!hbtp]\centering
	\includegraphics[width=0.9\linewidth]{../Graphs/defprob.pdf}
	\caption{Default probability}
\end{figure}

\section{Model with private savings}
We maintain the assumption that the government collects lump-sum taxes from the representative agent. But now the representative agent is also able to buy risk-free securities in international markets.

Because of the possibility of lump-sum taxation, the state variable for the economy is the sum of government-held and private-held `reserves.' When choosing the savings position of the economy, the government is bound by the private sector's Euler equation, except if the constraint on non-negative savings binds for the representative agent. Therefore,

\begin{align}
	V^R(b,a,\state) = \max_{b', a', c_T, h} &\; u(c_T, F(h)) + \beta \ex{\mathcal{V}(b',a',\state')} \\
	\text{subject to }
	& c_T + q_a a' + \kappa b = a + y_T(\state) + q_b(b',a',\state) (b' - (1-\delta) b) \notag\\
	& h \leq \mathcal{H}(c_T, \bar{w}) \notag\\
	& u_T(c_T, F(h)) q_a \geq \beta \ex{u_T(c_T', F(h'))}\notag
\end{align}

Similarly, the value in default is now
\begin{align}
	V^D(b,a,\state) = \max_{a',c_T, h} &\; u(c_T, F(h)) + \beta \ex{ \theta \mathcal{V}(b,a',\state') + (1-\theta) V^D(b,a',\state) } \\
	\text{subject to }
	& c_T + q_a a' = a + \mathcal{D}(y_T(\state)) \notag \\
	& h \leq \mathcal{H}(c_T, \bar{w}) \notag \\
	& u_T(c_T, F(h)) q_a \geq \beta \ex{u_T(c_T', F(h'))} \notag
\end{align}

\paragraph{Choice variables} As before, in equilibrium $h = \min\{1,\mathcal{H}(c_T, \bar{w})\}$. So a two-step procedure could be (i) find the set $A(b) = [a_{\min}(b), a_{\max}(b)]$ where $a$ can be chosen respecting the private Euler equation (using the previous formula and the budget constraint for $h$ and $c_T$) and (ii) choose $b$ and $\theta \in [0,1]$ (with $a = a_{\min}(b) + \theta (a_{\max}(b) - a_{\min}(b)$).

\paragraph{Decentralized formulation}
The government has reserves $a$, the representative agent has savings $k$. The government collects lump-sum taxes $T$. Their choices are summarized by
\begin{align}
	w^R(b,a,k,T,\state) = \max_{c_T, c_N, k'} &\; u(c_T, c_N) + \beta \ex{w(b',a',k',T',\state')} \\
	\text{subject to } 
	& c_T + q_a k' + p_N c_N = k + y_T(\state) + p_N y_N - T \notag
\end{align}
We obtain the following first-order conditions, where also applying market clearing conditions $c_N = y_N$
\begin{align*}
	p_N &= \frac{\varpi_N}{\varpi_T} \pfrac{c_T}{F(h)}^{1+\eta} \\
	u_T(c_T, F(h)) q_a &\geq \beta \ex{u_T(c_T', F(h'))}
\end{align*}
The intratemporal condition is the same as the planner's as this choice between consumption of both goods is undistorted in the equilibrium (the planner does not need a second tax instrument to match the wedge in the consumption allocation decision).

The planner's problem is
\begin{align}
	v^R(b,a,k,\state) = \max_{b',a',T} &\; u(c_T, F(h)) + \beta \ex{v(b',a',k',\state')} \\
	\text{subject to }
	& q_a a' + \kappa b = q_b b' + a + T \notag
\end{align}
\end{document}