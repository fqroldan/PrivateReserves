\documentclass[11pt,letterpaper]{article}
\usepackage[american]{babel}
\usepackage{amssymb,amsfonts,amsmath,amsthm,amscd,verbatim,latexsym,xspace,mathtools,array,pifont}
%\usepackage[sfdefault]{FiraSans}
\usepackage{ifxetex}
\ifxetex
	\usepackage[math]{mathspec}
	\setmainfont[Numbers=OldStyle]{Linux Libertine}
	\setmathsfont(Latin,Digits){Linux Libertine}
	\setlength{\parskip}{0.15cm}
	\linespread{1.3} % Line spacing
\else
	\usepackage[osf]{mathpazo}
	\usepackage{palatino}
\fi

\usepackage{booktabs}
\usepackage{datetime}
\usepackage{bbm} %for the sweet indicator function
\usepackage[usenames,dvipsnames]{xcolor}
\usepackage{enumerate}
\usepackage{graphicx}
\usepackage{etoolbox}
\usepackage{listings} % Required for insertion of code
%\usepackage{courier} % Required for the courier font
\usepackage{makeidx}
\usepackage[left=2.0cm,right=2.0cm,top=1in,bottom=1in,nomarginpar,bindingoffset=0cm,centering]{geometry}
%\usepackage{paralist} % Used for the compactitem environment which makes bullet points with less space between them
\usepackage{caption}
\usepackage{subcaption}
\captionsetup[table]{labelfont=sc,textfont=sc}
\captionsetup[figure]{labelfont=sc,textfont=sc}

\usepackage{titlesec} % Allows customization of titles
%\renewcommand\thesection{\Roman{section}} % Roman numerals for the sections
%\renewcommand\thesubsection{\Alph{subsection}} % Roman numerals for subsections
\titleformat{\section}[block]{\large\scshape\centering}{\thesection.}{1em}{} % Change the look of the section titles
\titleformat{\subsection}[block]{\large\itshape}{\thesubsection\xspace}{1em}{} % Change the look of the section titles

\usepackage{fancyhdr} % Custom headers and footers
\pagestyle{fancyplain} % Makes all pages in the document conform to the custom headers and footers
\fancyhead{} % No page header - if you want one, create it in the same way as the footers below
\fancyfoot[L]{} % Empty left footer
\fancyfoot[C]{} % Empty center footer
\fancyfoot[R]{\thepage} % Page numbering for right footer
\renewcommand{\headrulewidth}{0pt} % Remove header underlines
\renewcommand{\footrulewidth}{0pt} % Remove footer underlines
\setlength{\headheight}{10pt} % Customize the height of the header
\usepackage[authoryear]{natbib}

%\numberwithin{equation}{section}
%\numberwithin{table}{section}
\definecolor{nyupurple}{HTML}{57068C}
\definecolor{imfblue}{HTML}{004c97}
\definecolor{DarkTeal}{HTML}{08282e}

\newcommand{\dem}[2] {\bigskip \indent \textbf{Proof of #1 }#2 \hfill $\square$}
\newcommand{\cmark}{\ding{51}}%
\newcommand{\xmark}{\ding{55}}%
\def\subjectto{\quad \text{ subject to } \quad}
\newcommand{\real}{\mathbb{R}}
\newcommand{\nat}{\mathbb{N}}
\newcommand{\der}[2]{\frac{\partial #1}{\partial #2}}
\newcommand{\pfrac}[2]{\left( \frac{#1}{#2} \right)}
\newcommand{\prob}[2][{}]{\mathbb{P}_{#1}\left( #2 \right)}
\newcommand{\ex}[2][{}]{\mathbb{E}_{#1}\left[ #2 \right]}
\newcommand{\var}[2][{}]{\text{var}_{#1}\left( #2 \right)}
\newcommand{\ent}[1]{\text{ent}\left( #1 \right)}
\newcommand{\alert}[1]{{\color{red!75!black}#1}}
\newcommand{\note}[1]{\emph{Note:} {\color{red!60!black}#1}}
\newcommand{\noopsort}[2]{#2}
\newcommand{\intro}[1][{Introduction}]{\addtocounter{section}{+1} \section*{{#1}}}
\newcommand{\red}[1]{{\color{red} #1}}
\def\ind{\mathbbm{1}}
\def\mc{\mathtt{mc}}
\def\NFA{\mathtt{NFA}}
\def\nfa{\mathtt{nfa}}
\def\haircut{\hslash}
\newcommand{\horrule}[1]{\rule{\linewidth}{#1}}
\def\state{\text{\textbf{s}}}
\def\State{\text{\textbf{S}}}


\usepackage{tikz}
\usetikzlibrary{shapes,arrows}
\tikzstyle{block} = [rectangle, draw, fill=blue!20, 
    text width=10em, text centered, rounded corners, minimum height=4em]
\tikzstyle{line} = [draw, -latex']
\usetikzlibrary{arrows, decorations.markings}
      \tikzstyle{vecArrow} = [thick, decoration={markings,mark=at position
      1 with {\arrow[semithick]{open triangle 60}}},
      double distance=1.4pt, shorten >= 5.5pt,
      preaction = {decorate},
      postaction = {draw,line width=1.4pt, white,shorten >= 4.5pt}]
     \tikzstyle{innerWhite} = [semithick, white,line width=1.4pt, shorten >= 4.5pt]
\usetikzlibrary{positioning}

\newcommand*\mycirc[1]{%
	\begin{tikzpicture}[baseline=(C.base)]
		\node[draw,circle,inner sep=1pt](C) {#1};
	\end{tikzpicture}}

\newtheorem{lemma}{Lemma}

\usepackage[pdfstartview=FitH]{hyperref}
\hypersetup{
    colorlinks	= true,
    linkcolor	= blue!75!black,
    urlcolor	= blue!50!black,
    citecolor 	= blue!75!black
}


\makeatletter
\patchcmd{\@bibitem}{\ignorespaces}{\label{bib-#1}\ignorespaces}{}{}\makeatother
\newcommand{\mycite}[1]{[\ref{bib-#1}]}
\newcommand{\myqcite}[2]{[\ref{bib-#2}, #1]}
\newcommand{\mytcite}[1]{\texorpdfstring{\cite{#1}}{\mycite{#1}}}
\newcommand{\mytqcite}[2]{\texorpdfstring{\citep[#1][]{#2}}{\myqcite{#1}{#2}}}